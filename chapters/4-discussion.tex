\section{Discussion}

There is an increasing number of research groups studying and developing CAD of
prostate cancer. The topic was recently reviewed by Lemaître et
al.\ \cite{Lemaitre2015}. Most of the publications have focused on the task of
cancer detection rather than characterization. However, both are required for
proper treatment decision planning. In this study, we built a classifier for
prostate cancer characterization utilizing texture features extracted from T₂w,
the monoexponential and kurtosis models of high-b-value DWI, and T₂ maps.

Several studies have demonstrated correlation between ADCₘ values and Gleason
score based on biopsy \citep{Turkbey2011, Tamada2008} or prostatectomy specimens
\citep{Toivonen2015, Peng2013, Boesen2015, Rosenkrantz2015, Donati2014}. Most of
the studies evaluating the performance of DWI for Gleason score prediction have
used first-order statistical features, which do not consider the spatial
relationships between voxels. Analyzing the texture may add useful information
regarding tumor heterogeneity and other structural properties. In our previous
studies we observed rectangular, fixed-shape regions-of-interest, and each tumor
was characterized by only one variable per image type \citep{Toivonen2015,
Jambor2015Relaxation, Jambor2015Rotating}. However, in this study we measured
the texture properties of each of the image types (T₂w, ADCₘ, ADCₖ, K, T₂).
We have shown that the characterization performance of prostate cancer can be
improved by combining texture features from the monoexponential and kurtosis
models, and the T₂w.

Most studies on texture analysis of PCa include only a small number of texture
descriptors and configurations \citep{Kwak2015, Viswanath2012, Ginsburg2014}. In
this study, we utilized a large number of both, from multiparametric source.
This allows evaluating a huge number of feature combinations, as the
regularization prevents overfitting caused by high dimensionality.

Texture analysis of multiparametric MRI has previously seen limited use in PCa
characterization. Fehr et al.\ \cite{Fehr2015} evaluated PCa characterization
with the whole-lesion first order statistics and GLCM texture features from a
similar-size dataset of ADC and T₂w. They used oversampling to ward off effect
of class imbalance. Similarly to our study, they integrated dynamic feature
selection as part of the training (using the recursive feature selection support
vector machine, RFE-SVM). In our study, we included a much more diverse and
numerous set of features, as one of our goals was to evaluate various texture
extraction methods. Moreover, we have for the first time demonstrated that using
texture features from K (kurtosis function) provided improvements to ADCₘ
(monoexponential function). This is important since first order statistics of
parameters derived from kurtosis function do not lead to improved performance of
ADCm (monoexponential function). The effect of noise remains to be explored in
future studies.

Tiwari et al.\ \cite{Tiwari2013} classified PCa using GLCM and simple gradient
features from T₂w and MR spectroscopy (MRS). A multi-kernel classifier with
graph embedding was used to reduce dimensionality. Compared to the current
study, they had fewer patients, and the classification was done on
equally-sized, rectangular metavoxels.

Furthermore, Wibmer et al.\ \cite{Wibmer2015} studied the associations of
Gleason scores and individual GLCM features from ADC and T₂w of PZ lesions,
using generalized linear regression and generalized estimating equations; and
Vignati et al.\ \cite{Vignati2015} tested Gleason score differentiation using
two of the GLCM features (contrast and homogeneity) from T₂w and ADC
individually.

Contrarily to previous approaches to performing non-rigid deformation and
co-registration of datasets with subsequent resampling to common space and
resolution \citep{Viswanath2012, Ginsburg2014}, in the current study the
prostate and tumor masks were done for each MR imaging method (T₂w, DWI, T₂)
individually, allowing us to perform texture analyses at their original native
resolutions. Estimating the effect of co-registration and resampling on texture
extraction is not trivial, and the process could cause loss of information.
However, the accuracy of the delineations in this study could be potentially
improved by an added step of co-registering MRI to histology images
\citep{Bourne2017}.

We highlight the limitation of performing re-slicing and
non-rigid deformation of MR data sets to common space and then co-registering
with whole mount prostatectomy sections. As noted by Bourne et
al.\ \cite{Bourne2017}, co-registration of whole mount prostatectomy sections to
MRI data sets is important. However, the effect of re-slicing and non-rigid
deformation of MR data sets to common space remains to be explored.

The gray-level co-occurrence matrix may well be the most widely used tool for
texture analysis of prostate MRI data sets. The Sobel operator, Gabor filters,
Haar transform, and local binary patterns have already been extensively applied
for texture analysis of prostate MRI, as have a few others~\citep{Lemaitre2015}
not included in this study.

The image moments, on the other hand, have been used more often for global
morphological analysis like shape recognition rather than local texture
analysis, although they have been used for texture as well~\citep{Tuceryan1994,
Laws1980, Tuceryan1990}. To the best of our knowledge, there are no published
studies using moment-based texture analysis for detection or characterization of
prostate cancer using MRI data sets. Tahmasbi et al.\ \cite{Tahmasbi2011} used
Zernike moments to characterize breast cancer, but as a global mass descriptor
and not for texture. Our results suggest that moment-based texture features
might be valuable for PCa characterization. More specifically, the best 1\%
features of the image types K and T₂, and the final model ADCₘ, K, and T₂w
combined included some of the texture features based on Hu or Zernike moments.

The GLCM summarizes pixel intensity occurrences, the Gabor descriptors detect
gradients of certain frequencies, and the LBP responds to point-like intensity
transformation patterns. The image moments describe the mass distribution of the
image content which is seen as a function that is integrated over space. Given
the supposed difference in tissue heterogeneity, it makes sense that a metric
based on mass distribution would discriminate lesions of varying Gleason scores.

Most of the texture extraction methods in this study use the sliding window
algorithm with seven or nine different window sizes depending on image
resolution. Usually, the window should be large enough to provide reliable
statistical information about its contents to characterize the texture, yet
small enough so that patches of different classes do not overlap too much
\citep{Haralick1973, Clausi2002Analysis}. The nature of each texture extraction
algorithm also affects the specific role and usefulness of each window
configuration. The optimal window size depends on method and data, and typically
cannot be estimated in practice without experimentation \citep{Puig2001}. Most
of the previous studies have utilized a very small number of different window
sizes, often without presenting validation for the choice. In this study, we
explored several window sizes simultaneously. This approach greatly increases
the number of features, which is usually something to be avoided in order to
produce an effective classifier. However, the machine learning method we used
scales well to a large number of features.

In addition to texture features, shape descriptors might provide information
useful for Gleason score characterization \citep{Hoeks2011}. However, we decided
to leave them out of this study, focusing on texture features only. Including
shape features would have required considerable amount of additional work, and
would make it more important to treat lesions of different prostate regions
differently.

We have evaluated an extensive number of MRI texture features in multivariate
setting for their ability to predict the Gleason score of prostate cancer.
Moreover, we have presented a machine learning system that, from a very large
number of candidate features, searches for a relevant subset for the task and
alternatively weights the features accordingly.

The single feature with highest
prediction performance estimate (AUC = 0.84) was a gray-level co-occurrence
matrix homogeneity of T₂w. The Gabor transform features performed well with the
ADC and K parameters. The lowest percentile statistics were useful with ADC and
T₂w. The features based on Hu and Zernike moments performed well for K and T₂.
Our results imply that a specific set of features and feature extraction methods
is needed to obtain maximal information from DWI, T₂w, and T₂.

The highest
overall performance estimate (AUC = 0.88) was obtained for the model utilizing a
small subset of texture features from the ADCₘ, K, and T₂w parameters. These
features included texture descriptors based on gray-level co-occurrence matrix,
Gabor transform, and the Zernike and Hu moments.

Our study has several limitations. First of all, only 62 patients were included
and further validation of our results in large patient cohort is needed. All of
the patients had gone through prostatectomy, and therefore it is biased on the
high Gleason score group with 80\% of the lesions. As is the case with many
previous studies, only one MRI data set per imaging method per patient was
evaluated. Therefore, the repeatability of the texture features cannot be
evaluated. Ideally, quantitative imaging methods would have high reliability and
repeatability, allowing the use of derived features for disease characterization
\citep{Shrout1979}.

Many of the texture extraction methods used in this study could be further
refined. Variations of the methods and the derived features have been proposed,
for example for Gabor filters \citep{Clausi2000}, and local binary patterns
\citep{Guo2012, Maenpaa2003}. For Gabor filters, schemes for unsupervised tuning
of optimal parameters have been proposed \citep{Teuner1995}. The Zernike moments
can be provided scale and transformation invariance \citep{Khotanzad1990}.

In the cross-validation process the set of selected features was slightly
different in every round, implying that some of the features may convey similar
information. This is natural since we tested such a large number of feature
candidates.

In this study, we focused on the characterization of histologically confirmed
and manually delineated cancer lesions. In a more practical setting, this
process should be preceded by automatic segmentation of the prostate and
detection of cancerous tissue. This limitation should be addressed in future
studies.

Studies show increased risk of PCa specific mortality for Gleason score 4+3 in
comparison to 3+4~\citep{Wright2009}. Differentiating these scores in the
characterization process would be useful in addition to the 3+3 vs >3+3
threshold that was considered in this study.

Our results suggest that the use of texture features extracted from T₂w, ADCₘ,
and K parametric maps leads to improved PCa characterization accuracy compared
to the more commonly used statistical features of DWI. In contrast, adding
features from T₂ did not improve the classification accuracy. The results
point out certain features and feature combinations that were succesful, out of
a very numerous set that includes various source image types, texture extraction
methods, window sizes, and method-specific configurations. Most of the useful
methods have already performed well in other studies (GLCM, Gabor, LBP).
However, the image moment based texture features (Hu, Zernike) appear to be
novel in the context of PCa characterization.
