%\section*{Abstract}

\subsection*{Purpose}

To develop and validate a classifier system for prediction of prostate cancer (PCa)
Gleason score (GS) using radiomics and texture features of T₂-weighted imaging (T₂w), diffusion
weighted imaging (DWI) acquired using high b values, and T₂-mapping (T₂).


\subsection*{Methods}

T₂w, DWI (12 b values, 0--2000 s/mm²), and T₂ data sets of 62 patients with
histologically confirmed PCa were acquired at 3T using surface array
coils. The DWI data sets were post-processed using monoexponential and kurtosis
models, while T₂w was standardized to a common scale. Local statistics and 8
different radiomics/texture descriptors were utilized at different configurations to
extract a total of 7105 unique per-tumor features. Regularized logistic
regression with implicit feature selection and leave pair out cross validation
was used to discriminate tumors with 3+3 vs >3+3 GS.


\subsection*{Results}

In total, 100 PCa lesions were analysed, of those 20 and 80 had GS of 3+3 and >3+3, respectively. The best model performance was obtained by selecting the top 1\% features of
T₂w, ADCₘ and K with ROC AUC of 0.88 (95\% CI of 0.82--0.95). Features from T2 mapping provided little added value. The most useful
texture features were based on the gray-level co-occurrence matrix, Gabor
transform, and Zernike moments.


\subsection*{Conclusion}

Texture feature analysis of DWI, post-processed using monoexponential and kurtosis models, and T₂w demonstrated good classification performance for GS of PCa. In multisequence setting, the optimal radiomics based texture extraction methods and parameters differed between different image types.
