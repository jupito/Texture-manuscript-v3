%\section*{Abstract}

\subsection*{Purpose}

To develop and validate a classifier system for prediction of prostate cancer
Gleason score using texture features of T₂-weighted imaging (T₂w), diffusion
weighted imaging (DWI), and T₂-mapping (T₂).


\subsection*{Methods}

T₂w, DWI, and T₂ data sets of 62 patients with histologically confirmed prostate
cancer were acquired at 3T using surface array coils. The DWI data sets were
post-processed using monoexponential and kurtosis models, while T₂w was
standardized to a common scale. Local statistics and 8 different texture
descriptors were utilized at different configurations to extract a total of
7105 unique per-tumor features. Regularized logistic regression with implicit
feature selection and leave pair out cross validation was used to discriminate
tumors with 3+3 vs >3+3 Gleason scores.


\subsection*{Results}

The best model performance was obtained by selecting the top 1\% features of
T₂w, ADCₘ and K with ROC AUC of 0.88 (95\% CI of 0.82--0.95). The most useful
texture features were based on the gray-level co-occurrence matrix, Gabor
transform, and Zernike moments.


\subsection*{Conclusion}

Texture feature analysis of DWI and T₂w hold promise for improved non-invasive
Gleason score prediction. In multiparametric setting, the optimal texture
extraction methods and parameters differ by image type.
