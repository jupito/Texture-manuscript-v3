\section{Introduction}

Prostate cancer (PCa) is the most common cancer in men, and the second most
common among the causes of death related to cancer. For example in USA, 161 360
new cases of PCa are estimated to be diagnosed in 2017, while the estimated
cancer related deaths are 26 730 \citep{Siegel2017}. However, in approximately
half of the cases of newly diagnosed PCa, the patients have a low risk of death
from the disease \citep{Walsh2007, Draisma2003}. Due to this wide range of
possible outcomes, it is important to accurately predict the risk caused by
PCa and to stratify patients accordingly aiming to limit over-treatment and
PCa mortality simultaneously.

The Gleason score is a commonly used marker for estimating the possible threat
posed by a PCa lesion, based on histopathological analysis of biopsy and
prostatectomy specimens under the microscope \citep{Epstein2005}. It is a two or
three-component numerical grading that is based on observed structural patterns,
and can be expected to provide indication of tissue abnormality and tumor's
estimated likeliness to spread (metastatic potential). Gleason score can be
estimated using speciments acquired by using transrectal ultrasound (TRUS)
guided prostate biopsy. Unfortunately, in 30--50\% of patients the findings
based on systematic TRUS do not represent true Gleason score \citep{Nepple2009,
Steinberg1997, Rajinikanth2008}.

Magnetic resonance imaging (MRI) is increasingly being used for the detection of
PCa lesions. Diffusion weighted MR imaging (DWI) has been shown to have
potential for the detection and characterization of PCa. DWI data sets are
still usually being analyzed by measuring only the first-order statistical
properties found in parametric maps such as apparent diffusion coefficient of
the monoexponential function (ADCₘ) \citep{Turkbey2011, Toivonen2015,
Jambor2015Relaxation}.

Various different fitting methods have been applied for modeling PCa DWI signal
decay. The biexponential function \citep{Mulkern2006} provides the best fitting
quality for DWI data sets acquired using ``high'' b values ($\sim$2000 s/mm²)
\citep{Jambor2015Evaluation}. The biexponential function is not robust against
noise and has low repeatability. In contrast, kurtosis function
\citep{Jensen2005} provides similar fitting quality while being substantially
more robust against noise \citep{Jambor2015Evaluation}. However, texture
features of parametric maps derived from the kurtosis function have not been
evaluated and hold promise by better signal characterization compared with the
most commonly used monoexponential function \citep{Toivonen2015}. Other MRI
methods, such as T₂-mapping (T₂) and anatomical T₂-weighted imaging (T₂w), could
provide complimentary information to DWI for prediction of PCa characteristics
\citep{Jambor2015Relaxation}.

Computer-aided diagnostics (CAD) based on MRI has been demonstrated to have
complementary role to a reporting radiologist in PCa detection \citep{Kwak2015,
Viswanath2012, Ginsburg2014}. However, only a limited number of studies focused
on characterizing the detected PCa lesions, and they typically utilize only a
small number of texture features. \citet{Tiwari2013} used MR spectroscopy data
and various texture features from T₂w to first detect PCa and then predict its
Gleason score. \citet{Peng2013} evaluated histogram-based features from
multiparametric MRI regarding correlation with Gleason score. Texture features
from DWI and T₂w have been assessed for differentiating Gleason scores
\citep{Wibmer2015, Vignati2015, Fehr2015}. \citet{Rozenberg2016} evaluated
whole-lesion histogram and texture features from DWI in order to predict Gleason
score upgrade after radical prostatectomy.

The aim of this study was to use carefully optimized high quality MRI data sets
to develop and validate machine learning methods for non-invasive Gleason score
prediction, meaning prediction of PCa aggressiveness. In this study, we built
and evaluated a classifier system based on multiple texture features of high
quality T₂w, DWI (monoexponential and kurtosis functions), and T₂ relaxation
maps for prediction of PCa Gleason score dichotomized as 3+3 (low risk) vs >3+3
(high risk). Moreover, we explored which combinations of imaging modalities and
texture extraction methods are most useful for Gleason score prediction.

The current study is first of its kind in multiple aspects: a) first time direct
comparison of textures extracted from T₂w and T₂ relaxation maps was performed,
b) first time elevation of texture features from non-monoexponetial DWI signal
decay using high quality data sets, and c) first time evaluation of large number
of texture features and methods to calculate them in a multi-dimensional high
quality MRI data sets of patients with PCa. Above all, free public access to all
imaging data sets and post-processing code will be provided following
publication.
