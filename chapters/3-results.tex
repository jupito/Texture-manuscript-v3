\section{Results}

The highest ranking features for differentiating Gleason scores are summarized
by texture extraction method in Table~\ref{tab:texture_best1p}. Features based
on Gabor filters were included in all image types. GLCM features were selected
for T₂w and T₂, and Zernike moments for K and T₂. Features from the Hu
moments and LBP were also selected for T₂, in which the top 1\% had more
variability than other image types, regarding both the texture extraction method
and window size.


\subsection{Univariate Analysis}

ROC analysis was performed for each texture feature. The resulting best features
are shown in Table~\ref{tab:texture_imagetype}. The best one was MBB-GLCM
homogeneity in T₂w with AUC of 0.84.

A similar analysis was performed for the first-order statistical features;
results are shown in Table~\ref{tab:stats_imagetype}. Although the best
statistical features had good performance for most of the modalities, they did
not out-perform the best texture features.

Some of the high-ranking features are visualized in Figure~\ref{fig:tmap}. In
addition, ROC curves for best statistical and texture features are presented in
Figure~\ref{fig:roc}.


\subsection{Multivariate Analysis}

Regularized logistic regression together with LPOCV was used to estimate the
prediction performance of models trained on extracted features. Following are
the results obtained when all features from a single image type were used and
also for the best selected features within image type
(Table~\ref{tab:auc_imagetype}). Moreover, performances of the models combining
features are presented in Table~\ref{tab:auc_combinations}.


\subsection{All features of individual image type}

When using all features and L1 regularization (Table~\ref{tab:auc_imagetype}),
T₂w had AUC (95\% CI) value of 0.82 (0.72--0.92), DWI derived parametric maps
(ADCₘ, ADCₖ, K) had AUC (95\% CI) values ranging from 0.64 (0.52--0.77) to
0.71 (0.58--0.83), and T₂ derived features had AUC (95\% CI) value of 0.58
(0.45--0.71). In contrast, logistic regression utilizing L2 regularization
showed better performance than using L1 regularization for all image types
except for T₂w where AUC (95\% CI) value dropped to 0.68 (0.55--0.82), DWI
derived parametric maps (ADCₘ, ADCₖ, K) had AUC (95\% CI) values ranging from
0.69 (0.57--0.81) to 0.73 (0.60--0.85), and T₂ derived features had AUC (95\%
CI) value of 0.70 (0.59--0.82). These results indicate that an L2 regularized
logistic regression model with all features might perform better than a logistic
regression model with fewer features selected by L1 regularization for all image
types except T₂w. However, it should be noticed that with T₂w L1
regularization performed better than L2, suggesting that a subset of features
would performed better than all of them.


\subsection{Selected features of individual image type}

The feature selection was based on filtering features by AUC\@. Only 1\% best
features (a total of 12 or 16 features depending on the modality) with highest
ranking AUC were selected in each image type. When using L1 regularization, the
best T₂w features showed better performance than the features of DWI
parametric maps (ADCₘ, ADCₖ, K). Specifically, AUC (95\% CI) for ADCₘ, ADCₖ,
and K were ranging from 0.71 (0.60--0.82) to 078 (0.67--0.89) while T₂w AUC
(95\% CI) value was 0.80 (0.69--0.90). The best T₂ features had AUC (95\% CI)
value 0.51 (0.37--0.65) which is the lowest performance among the modalities.

The estimated AUC values using L2 regularization and the best 1\% features,
compared with L1 regularization, were lower for T₂w and K.

The use of texture features of the individual image type did not substantially
out-perform predictions made by models utilizing the 18 statistical features of
the corresponding image type (Table~\ref{tab:auc_imagetype}), except in T₂w
where the texture feature model obtained with L1 regularization had the best
performance among all other models. The highest AUC (95\% CI) value based on
statistical features was achieved using L1 regularization and ADCₘ, 0.79
(0.68--0.90), while the corresponding value for the best 1\% texture features
was 0.71 (0.60--0.82).

The AUC values for the best 1\% texture features based on T₂w, ADCₘ, ADCₖ, K
and T₂ are shown in supporting material Tables S2, S4, S6, S8, and S10,
respectively. Similarly, the statistical feature based AUC values for each image
type are shown in supporting material Tables S3, S5, S7, S9, and S11.


\subsection{All features of combined image types}

No substantial improvements of AUC values were present when combining all
features of image types (Table~\ref{tab:auc_combinations}). The AUC (95\% CI)
values were in the range from 0.53 (0.40--0.66) to 0.82 (0.73--0.91) for L1
regularization and from 0.69 (0.56--0.81) to 0.80 (0.71--0.89) for L2
regularization.


\subsection{Selected features of combined image types}

In contrast to the use all features, a better model performance was present when
combining the best 1\% features of various image types (T₂w, ADCₘ, ADCₖ, K,
T₂). The best model performance was obtained when selecting the best 1\%
features based on ADCₘ, K and T₂w with AUC (95\% CI) value of 0.88
(0.82--0.95). The combinations of features extracted from DWI parametric maps
(ADCₘ, ADCₖ, K) and those extracted from T₂w and T₂ together with the
feature selection method lead to improved prediction performance regardless of
the regularization algorithm (L1, L2).

The final model proposed to differentiate PCa low Gleason score and high Gleason
score includes the features from ADCₘ, K and T₂w (listed in S12) and the
expected performance of the model is showed in Figure~\ref{fig:roc}.
